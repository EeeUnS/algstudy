
\documentclass[10pt]{beamer}
\usepackage{kotex}

\usepackage{framed}
\usepackage{graphicx}
%https://www.overleaf.com/learn/latex/Inserting_Images

\usepackage{amsmath}
%use dfrac
\usepackage{xcolor}

\usepackage{amsthm}
%\usepackage{tabl}
\usepackage{listings}
\definecolor{mGreen}{rgb}{0,0.6,0}
\definecolor{mGray}{rgb}{0.5,0.5,0.5}
\definecolor{mPurple}{rgb}{0.58,0,0.82}
\definecolor{backgroundColour}{rgb}{0.95,0.95,0.92}
%https://tex.stackexchange.com/questions/348651/c-code-to-add-in-the-document
\lstdefinestyle{CppStyle}{
    backgroundcolor=\color{backgroundColour},   
    commentstyle=\color{mGreen},
    keywordstyle=\color{magenta},
    numberstyle=\tiny\color{mGray},
    stringstyle=\color{mPurple},
    basicstyle=\footnotesize,
    breakatwhitespace=false,         
    breaklines=true,                 
    captionpos=b,                    
    keepspaces=true,                 
    numbers=left,                    
    numbersep=5pt,                  
    showspaces=false,                
    showstringspaces=false,
    showtabs=false,                  
    tabsize=2,
    language=C++
}

\usepackage{url}

\usepackage{etoolbox}
\AtBeginEnvironment{quote}{\singlespacing\small}


\usepackage{thmtools}
\usepackage{xcolor}
\declaretheoremstyle[% spaceabove=6pt,spacebelow=6pt, headfont=\color{MainColorOne}\sffamily\bfseries, notefont=\mdseries, notebraces={[}{]}, bodyfont=\normalfont,
headpunct={},
postheadspace=1em,
%qed=▣,
]{maintheorem}

\declaretheorem[%
name=정의,
style=maintheorem,
numberwithin=section, shaded={%bgcolor=MainColorThree!20,
margin=.5em}]{dfn}
% \begin{dfn}[]
% \end{dfn}

\setbeamertemplate{footline}[frame number]

\usetheme{Hannover}
%\usetheme{CambridgeUS}


\title{시간복잡도 기초}

\author{EUnS}

\begin{document}


\begin{frame}{}
    \maketitle
\end{frame}    

% \begin{frame}{}
%     \tableofcontents
% \end{frame}   

\begin{frame}{}
    \href{}{\textcolor{blue}{참고}}
\end{frame}    


\begin{frame}{$O$}
    \begin{itemize}
        \item $O(g(n)) = \{ f(n)$ 모든 $n \ge n_0$에 대해 $0 \le f(n) \le cg(n)$ 인 양의 상수 $c, n_0$이 존재한다. $\}$
        \pause
        \item $\Omega(g(n)) = \{ f(n)$ 모든 $n \ge n_0$에 대해 $0 \le  cg(n) \le f(n)$ 인 양의 상수 $c, n_0$이 존재한다. $\}$
        \pause
        \item $\Theta(g(n)) = \{ f(n)$ 모든 $n \ge n_0$에 대해 $0 \le  c_1 g(n) \le f(n) \le c_2g(n)$ 인 양의 상수 $c_1, c_2, n_0$이 존재한다. $\}$
    \end{itemize}
\end{frame}


\begin{frame}{}
    \begin{figure}[h!]
        %\centering
        \includegraphics[scale=0.25]{}
        \caption{}
    \end{figure}
\end{frame}    




\begin{frame}{$\Theta$}
    \begin{itemize}
        \item 
        \item 
    \end{itemize}
\end{frame}


\begin{frame}{$\Omega$}
    \begin{itemize}
        \item 
        \item 
    \end{itemize}
\end{frame}


\begin{frame}{$\Omega$}
    \begin{itemize}
        \item 
        \item 
        \item 
    \end{itemize}
\end{frame}


\begin{frame}[fragile]{}
        
    \begin{lstlisting}[style = CppStyle]
    
    \end{lstlisting}

\end{frame}    


\begin{frame}{}
    \begin{figure}[h!]
        %\centering
        \includegraphics[scale=0.25]{}
        \caption{}
    \end{figure}
\end{frame}    

\end{document}

% \stop


\begin{frame}{$O$}
    \begin{itemize}
        \item 
        \item 
        \item 
    \end{itemize}
\end{frame}
