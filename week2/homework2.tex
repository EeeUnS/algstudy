\documentclass{oblivoir}

\usepackage{hyperref}
\usepackage{xcolor}

\begin{document}
   \section{}
   
   insertion sort 짜고 \href{https.//www.acmicpc.net/problem/2750}{\textcolor{blue}{문제}} 풀기

   
   \section{}
   
   최대 부분 배열문제 분할정복 코드를 짜고 문제 풀기

   \begin{itemize}
      \item \href{https.//www.acmicpc.net/problem/1912}{\textcolor{blue}{문제}}
      
      \item \href{https.//www.acmicpc.net/problem/10211}{\textcolor{blue}{문제}}
   \end{itemize}
   

   \section{}
   
   Use the following ideas to develop a nonrecursive, linear-time algorithm for the maximum-subarray problem. 
   Start at the left end of the array, and progress toward the right, keeping track of the maximum subarray seen so far. 
   Knowing a maximum subarray of$ A[1.. j]$, extend the answer to find a maximum subarray ending at index $j+1$ by using the following observation. a maximum subarray of$ A[1.. j+1]$
is either a maximum subarray of $A[1.. j]$ or a subarray $A[i.. j + 1]$, for some
$1 \le i \le j + 1$. Determine a maximum subarray of the form $A[i.. j + 1]$ in
constant time based on knowing a maximum subarray ending at index $j$.

   최대 부분 배열 문제에대한 $\Theta(n)$ 알고리즘인 KADANE'S ALGORITHMS을 찾아보고 코드로 짜서 문제를 풀어보기
   
\end{document}