
\documentclass[10pt]{beamer}
\usepackage{kotex}

\usepackage{framed}
\usepackage{graphicx}
%https://www.overleaf.com/learn/latex/Inserting_Images

\usepackage{amsmath}
%use dfrac
\usepackage{xcolor}

\usepackage{amsthm}
%\usepackage{tabl}
\usepackage{listings}
\definecolor{mGreen}{rgb}{0,0.6,0}
\definecolor{mGray}{rgb}{0.5,0.5,0.5}
\definecolor{mPurple}{rgb}{0.58,0,0.82}
\definecolor{backgroundColour}{rgb}{0.95,0.95,0.92}
%https://tex.stackexchange.com/questions/348651/c-code-to-add-in-the-document
\lstdefinestyle{CppStyle}{
    backgroundcolor=\color{backgroundColour},   
    commentstyle=\color{mGreen},
    keywordstyle=\color{magenta},
    numberstyle=\tiny\color{mGray},
    stringstyle=\color{mPurple},
    basicstyle=\footnotesize,
    breakatwhitespace=false,         
    breaklines=true,                 
    captionpos=b,                    
    keepspaces=true,                 
    numbers=left,                    
    numbersep=5pt,                  
    showspaces=false,                
    showstringspaces=false,
    showtabs=false,                  
    tabsize=2,
    language=C++
}

\usepackage{url}

\usepackage{etoolbox}
\AtBeginEnvironment{quote}{\singlespacing\small}


\usepackage{thmtools}
\usepackage{xcolor}
\declaretheoremstyle[% spaceabove=6pt,spacebelow=6pt, headfont=\color{MainColorOne}\sffamily\bfseries, notefont=\mdseries, notebraces={[}{]}, bodyfont=\normalfont,
headpunct={},
postheadspace=1em,
%qed=▣,
]{maintheorem}

\declaretheorem[%
name=정의,
style=maintheorem,
numberwithin=section, shaded={%bgcolor=MainColorThree!20,
margin=.5em}]{dfn}
% \begin{dfn}[]
% \end{dfn}

\setbeamertemplate{footline}[frame number]

\usetheme{Hannover}
%\usetheme{CambridgeUS}


\title{확률적 분석}



\author{EUnS}

\begin{document}


\begin{frame}{}
    \maketitle
\end{frame}    

% \begin{frame}{}
    %     \tableofcontents
% \end{frame}   


\subsection{Indicator random variables} 

\begin{frame}{평균 수행시간}
    \begin{itemize}
        \item 입력 값의 분포에 대한 수행시간의 평균값.
        \item $\Pr$ : 확률
        \item $\displaystyle E[X] = \sum_{x=1}^n x\Pr\{X=x\}$
    \end{itemize}
\end{frame}

\begin{frame}{고용 문제}
    \begin{itemize}
        \item 매일 한명씩 지원자가 와서 면접을 본다.
        \item 지원자가 현재 고용자보다 뛰어나면 해고하고 지원자를 고용한다.
        \item 이때 면접 비용 $c_i$와 고용 비용 $c_h$가 든다.
        \item 고용 비용이 면접 비용보다 훨씬 비싸다.
        \item 이때 면접과 고용에 드는 비용을 알고싶다.
    \end{itemize}
\end{frame}


\begin{frame}[fragile]{}
    \begin{lstlisting}[style = CppStyle]
    HIRE-ASSISTANT(n)
        best = 0 // candidate 0 is a least-qualified dummy candidate
        for i = 1 to n
            interview candidate i
            if candidate i is better than candidate best
                best = i
                hire candidate i
    \end{lstlisting}
\end{frame}    

\begin{frame}{}
    \begin{itemize}
        \item 고용된 인원을 $m$이라 할때 총 비용은 $O(c_in + c_hm)$
        \item 최악의 경우 고용비용
        \pause
        $O(c_hn)$
        \item 평균 고용 비용은?
    \end{itemize}
\end{frame}


\begin{frame}{지표확률변수}
    \begin{align*}
        I\left\{ H \right\} =  
    \begin{cases}
        1 &\mbox{( $H$ 발생)} \\
        0 &\mbox{( $\bar{H}$ 발생)}
    \end{cases}   
    \end{align*}
    확률 변수를 0,1로 고정한 특별한 확률변수 횟수를 셀때 유용
    \pause
    \[
        \begin{aligned}
        E[X_A] &= E[I\{A\}] \\  \pause
            &=1 \cdot \Pr\{A\} + 0 \cdot \Pr\{\bar{A}\}\\ \pause
            &= \Pr\{A\}        
        \end{aligned}
    \]
\end{frame}



\begin{frame}{}
    \begin{itemize}
        \item $X$ : 새로운 직원을 고용한 횟수에 대한 확률 변수.
        \item $X_i$ : $i$번째 지원자가 고용되었는지에 대한 지표 확률 변수.
    \end{itemize}

    \begin{align*}
       X_i  = I\left\{ \mbox{지원자 $i$가 고용됨} \right\} =  
    \begin{cases}
        1 &\mbox{(지원자 $i$ 고용)} \\
        0 &\mbox{(지원자 $i$가 고용 안됨)}
    \end{cases}    
    \end{align*}
    $X = X_1 + X_2 + \cdots + X_n$
\end{frame}



\begin{frame}{}
    \begin{itemize}
        \item $\Pr\{\mbox{지원자 $i$가 고용될 확률}\}$ ???
        \pause
        \item $E[X_i] = \dfrac{1}{i}$
        \pause
    \end{itemize}

    \[
        \begin{aligned}  
          E[X] &= E \left[  \sum_{i=1}^n X_i  \right] \\ \pause
          &= \sum_{i=1}^n E[X_i] \\ \pause
          &= \sum_{i=1}^n \dfrac{1}{i} \\ \pause
          &= \ln n + O(1)
        \end{aligned}
    \]
\end{frame}



\begin{frame}{결론}
    \begin{itemize}
        \item 평균 고용 비용 : $O(c_h \ln n)$
        \item 총 평균 : $O(c_in + c_n \ln n)$
    \end{itemize}

\end{frame}



% \begin{frame}{}
%     \href{}{\textcolor{blue}{참고}}
% \end{frame}    




% \begin{frame}{}
%     \begin{figure}[h!]
%         %\centering
%         \includegraphics[scale=0.25]{}
%         \caption{}
%     \end{figure}
% \end{frame}    


% \begin{frame}{$O$}
%     \begin{itemize}
%         \item 
%         \item 
%         \item 
%     \end{itemize}
% \end{frame}


\end{document}

% \stop

