
\documentclass[10pt]{beamer}
\usepackage{kotex}

\usepackage{framed}
\usepackage{graphicx}
%https://www.overleaf.com/learn/latex/Inserting_Images

\usepackage{amsmath}
%use dfrac
\usepackage{xcolor}

\usepackage{amsthm}
%\usepackage{tabl}
\usepackage{listings}
\definecolor{mGreen}{rgb}{0,0.6,0}
\definecolor{mGray}{rgb}{0.5,0.5,0.5}
\definecolor{mPurple}{rgb}{0.58,0,0.82}
\definecolor{backgroundColour}{rgb}{0.95,0.95,0.92}
%https://tex.stackexchange.com/questions/348651/c-code-to-add-in-the-document
\lstdefinestyle{CppStyle}{
    backgroundcolor=\color{backgroundColour},   
    commentstyle=\color{mGreen},
    keywordstyle=\color{magenta},
    numberstyle=\tiny\color{mGray},
    stringstyle=\color{mPurple},
    basicstyle=\footnotesize,
    breakatwhitespace=false,         
    breaklines=true,                 
    captionpos=b,                    
    keepspaces=true,                 
    numbers=left,                    
    numbersep=5pt,                  
    showspaces=false,                
    showstringspaces=false,
    showtabs=false,                  
    tabsize=2,
    language=C++
}

\usepackage{url}

\usepackage{etoolbox}
\AtBeginEnvironment{quote}{\singlespacing\small}


\usepackage{thmtools}
\usepackage{xcolor}
\declaretheoremstyle[% spaceabove=6pt,spacebelow=6pt, headfont=\color{MainColorOne}\sffamily\bfseries, notefont=\mdseries, notebraces={[}{]}, bodyfont=\normalfont,
headpunct={},
postheadspace=1em,
%qed=▣,
]{maintheorem}

\declaretheorem[%
name=정의,
style=maintheorem,
numberwithin=section, shaded={%bgcolor=MainColorThree!20,
margin=.5em}]{dfn}
% \begin{dfn}[]
% \end{dfn}

\setbeamertemplate{footline}[frame number]

\usetheme{Hannover}
%\usetheme{CambridgeUS}


\title{치환법, 마스터 정리}

\author{EUnS}

\begin{document}


\begin{frame}{}
    \maketitle
\end{frame}

% \begin{frame}{}
%     \tableofcontents
% \end{frame}   

\begin{frame}{치환법}
    \begin{enumerate}
        \item 해의 모양을 추측한다.
        \item 상수들의 값을 찾아내기 위해 수학적 귀납법을 사용하고 제대로 동작함을 보인다.
    \end{enumerate}
\end{frame}

\begin{frame}{치환법}
    $$T(n) = 2T(n/2) + n$$
    \begin{enumerate}
        \item $T(n) \le cn \lg n$이라고 추측
    \end{enumerate}
    \[
        \begin{aligned}
           T(n) &\le 2c(n/2\lg (n/2)) +n \\ \pause
            &= cn\lg n - n + n \\  \pause
            &\le cn\lg n
        \end{aligned}
    \]
\end{frame}


\begin{frame}{치환법}
    $$T(n) = 2T(n/2) + 1$$
    \begin{itemize}
        \item $T(n) = O(n)$이라고 추측
    \end{itemize}
    \[
        \begin{aligned}
           T(n) &\le 2c(n/2) +1 \\ \pause
            &= cn + 1
        \end{aligned}
    \]
    \begin{itemize}
        \item $T(n) \le cn $ X \pause
        \item $cn$보다 좀더 작은 값을 추측할것.
    \end{itemize}

\end{frame}


\begin{frame}{치환법}
    $$T(n) = 2T(n/2) + 1$$
    \begin{itemize}
        \item $T(n) \le cn - d$이라고 추측
    \end{itemize}
    \[
        \begin{aligned}
           T(n) &\le 2c(n/2)-d +1 \\ \pause
            &= cn -2d + 1 \\ \pause
            &\le cn - d
        \end{aligned}
    \]
\end{frame}


\begin{frame}{마스터 정리}
    $a(\ge 1)$와 $b(> 1)$가 상수고 $f(n))$이 양의 함수라 하고, 음이 아닌 정수에 대해 $T(n)$이 다음 점화식에 의해 정의된다고 하자.
    $$T(n) = aT(n/b) + f(n)$$
    여기서 $n/b$는 $\lfloor n/b \rfloor \lceil n/b \rceil$를 뜻한다. 이때 $T(n)$의 점근적 한계는 다음과 같다.
    \begin{enumerate}
        \item 상수 $\epsilon(>0)$에 대해 $f(n) = O(n^{\log_b a-\epsilon})$이면, $T(n) = \Theta(n^{\log_b a})$이다.
        \item $f(n) = \Theta(n^{\log_b a})$이면, $T(n) = \Theta(n^{\log_b a} \lg n)$이다.
        \item 상수 $c(<1)$와 충분히 큰 모든 $n$에 대해 $af(n/b) \le cf(n)$이면, $T(n) = \Theta(f(n))$이다.
    \end{enumerate}
\end{frame}



\begin{frame}{ex}
    $T(n) = 9T(n/3) + n$ \pause
    \begin{itemize}
        \item $a = 9$ \pause
        \item $b = 3$ \pause
        \item $f(n) = n = O(n ^{\log _3 9 - \epsilon})$ \pause
        \item case 1 \pause
        $$\Theta(n^2)$$
    \end{itemize}
     $T(n) = T(2n/3) + 1$ \pause
     \begin{itemize}
         \item $a = 1$ \pause
         \item $b = 3/2$ \pause
         \item $f(n) = 1 = \Theta(n ^{\log _{3/2}1}) = \Theta(1)$ \pause
        \item case 2 \pause
        $$\Theta(\lg n)$$
    \end{itemize}
\end{frame}

\begin{frame}{ex}
    $T(n) = 3T(n/4) + n\lg n$ \pause
    \begin{itemize}
        \item $a = 3$ \pause
        \item $b = 4$ \pause
        \item $c = 3/4$일 때 $af(n/b) = 3(n/4) \lg(n/4) \le (3/4)n \lg n = cf(n)$ \pause
        \item case 3 \pause
        $$\Theta(nlg n)$$
    \end{itemize}
\end{frame}


\begin{frame}{스트라센 알고리즘}
    $T(n) = 7T(n/2) + \Theta(n^2)$ \pause

    \begin{itemize}
        \item $a = 7$ \pause
        \item $b = 2$ \pause
        \item $f(n) = n^2 = O(n ^{\log _2 7 - \epsilon})$ \pause
        \item case 1 \pause
        $$\Theta(n^{lg 7})$$
    \end{itemize}
\end{frame}


\begin{frame}{적용할 수 없는 점화식}
    $T(n) = 2T(n/2) + n\lg n$ \pause
    \begin{itemize}
        \item $a = 2$ \pause
        \item $b = 2$ \pause
        \item $af(n/b) = 2(n/2) \lg(n/2) \le (3/4)n \lg n = cf(n)$ 인 $c$를 잡을 수 없음.
        \item case 3 X
        \item ???
        \item $f(n)$이 $n^{\log_b a}$보다 다항식적으로 크지않다면 구할 수 없음.
    \end{itemize}
\end{frame}



% \begin{frame}{}
%     \begin{figure}[h!]
%         %\centering
%         \includegraphics[scale=0.25]{}
%         \caption{}
%     \end{figure}
% \end{frame}    



\end{document}


\begin{frame}{$\Theta$}
    \begin{itemize}
        \item 
        \item 
    \end{itemize}
\end{frame}


\begin{frame}{$\Omega$}
    \begin{itemize}
        \item 
        \item 
    \end{itemize}
\end{frame}


\begin{frame}{$\Omega$}
    \begin{itemize}
        \item 
        \item 
        \item 
    \end{itemize}
\end{frame}


\begin{frame}[fragile]{}
        
    \begin{lstlisting}[style = CppStyle]
    
    \end{lstlisting}

\end{frame}    


% \begin{frame}{}
%     \begin{figure}[h!]
%         %\centering
%         \includegraphics[scale=0.25]{}
%         \caption{}
%     \end{figure}
% \end{frame}    

% \stop


\begin{frame}{}
    \href{}{\textcolor{blue}{참고}}
\end{frame}


\begin{frame}{$O$}
    \begin{itemize}
        \item 
        \item 
        \item 
    \end{itemize}
\end{frame}

